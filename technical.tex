\documentclass[12pt]{article}
\usepackage{fullpage}
\usepackage{times}

\bibliographystyle{plain}

\title{Creating a Scalable and Reliable Peer Assessment System for
  Mathematical Proofs}

\author{Randal E Bryant \\ Luis Von Ahn}

\begin{document}

\section*{Project Description}

This research seeks to devise and evaluate mechanisms that will enable
students to get useful and reliable assessments of the proofs they
write in mathematics and computer science classes from their
peers---both students currently enrolled in the class and those who
have taken the course recently and are now serving as graders.  It
also seeks to maximize the educational benefit to the students
performing the assessments, by teaching them how to critically review
and evaluate proofs.  Within the scope of this proposal, the work will
focus on supporting large classes, with enrollments of 200 or more students.
This is a step toward a longer term goal of supporting peer
assessment in web-scale courses, with 10,000 or more participants.

\section{Motivation}

Writing clear and concise proofs is a fundamental skill
in mathematics.  Writing a good proof requires the ability to
rigorously decompose an argument into a sequence of steps that lead
from a set of premises to
a desired conclusion.  It also requires the
writer to communicate to the reader enough insight and information to guide them
through the reasoning process at an appropriate level of detail.

Being able to write proofs is also a fundamental requirement in the
education of computer scientists.  Not only does it provide a
mechanism where they can formally verify that a program will function
correctly \cite{hoare-cacm69}, it also enables them to analyze the
time and space complexity of an algorithm or program.  Even though few
programmers prove theorems about their programs on a routine basis,
knowing how to do so sharpens their ability to systematically reason
about code as they create.  It also helps them reason through a
specific execution when they try to debug it, determining both how
they arrived at this point in the program and what changes must be
made to prevent an error from occuring.


Learning how to write good proofs can be very challenging for many
students.  In many ways, it is like learning to write a compelling
essay or to paint a beautiful picture---it requires internalizing a
core set of rules and technical skills, while also learning how to
communicate concepts to others.  Just as with creative writing or art,
the ideal learning environment for proof writing is one where
students get personal attention by expert mentors.

Unfortunately, most undergraduate students get their early training in
mathematics and computer science in large classes, where it is
impractical for the instructor to give each student extensive
individual guidance.  If students are to learn to write proofs
starting at an early point in their mathematical and computer science
education, there must be a system for providing them accurate and
timely assessment of their work even in the context of high enrollment
classes.  Such assessments are essential to help them improve their
abilities (formative) and to give them proper credit for their efforts
(summative.)

As a case in point, we recently revised the undergraduate computer science
curriculum at Carnegie Mellon University so that all majors and many
nonmajors take two programming courses---one on imperative programming
and one on functional programming---that have students writing and
proving preconditions, postconditions, and invariants
\cite{bryant-ugrad10}.  Students also take courses in
discrete mathematics and introductory theoretical computer science
that involve proofs and formal reasoning, all within their first two years.
These courses each have
total annual enrollments of 200--500 students.  We handle our
assessment needs with large teams of undergraduate graders, who are
among our best students, but they still lack extensive experience in
the subject material and in critically evaluating proofs.  Ensuring
reliable, consistent, and uniform grading in such environments has proved
challenging.

Going beyond the need to provide assessments for classes with
enrollments in the hundreds, consider the recent advent of {\em web-scale}
courses, such as Massively Open Online Courses (MOOCs),
 offered to 10,000 or more students for free over the Internet
\cite{carson-am12,lewin-nyt12}.
Although it is easy enough to provide simple
assessments, such as multiple-choice questions, to these students,
there is no feasible way to have students attempt to write proofs and
then get reliable feedback on their efforts.  This shortcoming
seriously reduces the degree to which the participants can gain a deep
understanding of the theoretical aspects of computer science.

In computer science terms, we require a system for providing
assessments of proofs that is {\em scalable}, meaning that it can meet
the quantitative and qualitative demands for assessments using the
limited resources available.  We consider scaling at both the
large-class and web-scale levels, but with 
resources limited to: 1) an instructor, 2) a small team of assistants,
who may range in talent from almost novices to near experts, and 3) the
students themselves, who are just learning the material and who have
almost no experience in critically evaluating proofs.

We approach this problem as one of {\em human computation}, using
computer technology to devise systems in which useful work can be done by
harnessing the power of large numbers of nonexperts
\cite{quinn-chi11, vonahn05}.  This requires
the task to be decomposed into smaller units of work, each of which 
can be performed by the available talent pool.  It also requires
mechanisms to generate these units, supply them to the workers, and
then aggregate the results, such that the outcome is valid, even when
some of the workers either inadvertently or maliciously misuse the system.

\section{Existing Work on Peer Assessment}

The published literature on teaching students how to write good proofs
and how to assess their work is remarkably sparse.  Most published
work focuses on creating a tightly knit group of students working
under the close supervision of an instructor
\cite{cohen-amm82,jones-amm77}, or a small-enrollment, seminar-style
course \cite{reisel-amm82}.  These environments do allow students to
discuss and evaluate each others work, but they cannot be scaled to
larger classes.  Indeed, it seems that most math programs avoid
proof-based assignments with their large-enrollment, introductory
courses.

The common approach for grading any assignment for a large
class with multiple graders is for the instructor to generate a {\em
  scoring rubric} giving a detailed description of what components the
solution must contain, how each of these components should be
evaluated, and how points should be assigned
\cite{moskal-pa2000}.

Perhaps the most relevant prior work was done by Zerr at the
University of North Dakota \cite{zerr-primus11}.
He instituted formal peer evaluation, but
in a way that actually increased the required effort by the
instructor, rather than reducing it.  Even so, it produced some
useful insights into the challenges of peer assessment.  They found
the students were good at identifying correct proofs and then
supplying useful advice on how to improve the presentation.  For
incorrect proofs, however, many of the reviewers failed to even detect
that they were incorrect, and even fewer were able to pinpoint the exact error.

Other computer scientists have created generic, online peer review systems
\cite{gehringer-sigcse05, wolfe-ite04}, but these did not address the
assessment of proofs explicitly, and indeed provided little or no
framework for guiding the assessment process.

\subsection{Previous Work at Carnegie Mellon}

At Carnegie Mellon University, one of our undergraduate graders, Adam
Blank, created a system that serves as the prototype for the proposed
research.  Indeed, he is now a PhD student and this proposal seeks
funding to support him.  The system is designed to support large classes,
with the assessment performed by the students in the class and by
undergraduate graders.  It has been used in classes with enrollments
ranging from 60 to 380 students, in both computer science (introduction
to theoretical computer science) and mathematics (discrete math).

Elements of this system demonstrate 
how a scalable and reliable peer assessment system can work.  First,
the task of assessing a proof is defined as one of identifying a set of
{\em attributes} that are applicable to a particular proof.  These are
positive and negative facts that may hold for the proof.  Example
attributes include:
\begin{itemize}
\item The proof is by induction on the size of the array.
\item The proof uses linearity of expectation.
\item The base case of the induction is correct.
\item The proof contains an arithmetic error.
\end{itemize}
The list of possible attributes for a problem is generated by the
instructor and by experienced graders, based on grading a sample of
around 15 proofs.  Of the 148 problems that have been graded with
attribute-based grading, nine had 100 or more applicable attributes,
while the average was 40.6.

The actual grade for a proof is computed using a formula, devised
by the instructor, based on which attributes apply.  This mechanism
nearly eliminates any subjective evaluation by the less experienced graders
or the student reviewers and ensures uniformity of the grading
standards across the team of graders.  Each proof is reviewed by 3--5
students and by one grader.  The task for the student reviewers is to
identify which attributes they believe apply to a proof, while for the
grader it is to filter through the proposed attributes to see which
ones really do apply.  Having multiple reviews increases the chance
that the set of possible attributes will be complete.  To avoid
collusion, students are required to identify their collaborators,
forming edges in an undirected graph with students as nodes.  The
set of students is partitioned according to the connected components
of the graph, and
reviewers are assigned randomly, but such that they never review
assignments generated by students in their own components.  Reviewers
are graded on a 0--2 point scale (0 = useless, 1 = somewhat helpful,
and 2 = accurate.) 

We can see how this prototype system implements a form of human computation,
enabling nonexperts to perform useful and reliable work.  The normally
subjective task of assessing a proof is put into a structured
framework, such that it can be decomposed into a number of simpler
tasks that can be performed by nonexperts.  Workers are assigned
different types of tasks according to their capabilities and trustworthiness.
Randomization, redundancy, and selectivity are used in the task assignment to
improve quality and to ensure integrity of the process.

Although the conditions for and scale of the deployment of our
prototype system were insufficient to perform a systematic evaluation,
we have gained some insights from anecdotal evidence and from a survey
conducted in one of the classes.  First, the system worked better than existing
rubric-based approaches.
The graders found that the student
reviewers generally identified all of the applicable attributes for a
proof, although their selections had to be pruned down and
occasionally augmented.  The graders found having this input from the
student reviewers useful, even though they still had to carefully
review the proofs themselves.  Second, the instructors found the
grades assigned to the proofs to generally be fair assessments of
their quality.  The idea of generating a score automatically from the
set of applicable attributes was seen as greatly improving the
uniformity of the grading, while also yielding appropriate scores.
Of course, we cannot claim these results without a more rigorous
evaluation, but they lend hope that the overall approach is viable.

A survey conducted of students in one class indicated that they had
mixed opinions about being required to participate in the reviewing
process.  The major complaint was that they were being asked to do
extra work in a course that was already perceived as requiring too much
work.  Even so, when asked whether they found performing the reviews
to be helpful to their understanding of the material, 42\% said yes,
28\% said no, and 30\% were neutral.  So, there is some justification
for claiming that having students perform peer reviews has an educational
value, but more must be done to maximize the real and perceived
educational benefit of doing reviews.

\section{Coping with Scale}

Most large classes, including those using our prototype system, use a
two-level hierarchy for assessing student work.  A single instructor
guides the work of a staff of graders, who then oversee the efforts of
a much larger groups of students.  One simple model for such a system
is to assume that a single person can monitor the work of around $k$
others.  Then a two-level hierarchy can have one instructor running a
class with $k$ staff members and $k^2$ students.  A typical value
range for $k$ might be 20--40, making it possible to have classes with
enrollments ranging from 400 to 1,600.  Although this model is
obviously simplistic, we can see some validation with very large
courses.  For example, Prof.~James Maas of Cornell University was
renowned for teaching an introductory psychology course with 1,600
students, staffed by 22 teaching assistants.\cite{arenson-nyt00}.
Harvard's introductory computer science course CS~50 has an enrollment
of around 600 students, and their web page lists over 50 staff
members.  In the former case, the graduate teaching assistants working
for Prof.~Maas could probably handle a heavier load than could the
undergraduates staffing the Harvard course.  We can also see more than
a simple two-level structure in the Harvard class, with some staff members
serving as teaching assistants and others as graders.

A two-level hierarchy suffices for almost any course at a single
institution, but it cannot grow to web scale, in that it would require
our factor $k$ to be 100 or more to staff a course with over 10,000
students.  One could institute a deeper hierarchy.  For $n$ students,
we would require $\log_k n$ levels of staffing.  This would require a
total staff of around $n/(k-1)$ people.  For example, for $k = 30$, a
three-level hierarchy could handle 27,000 students, but it would
require over 900 staff members.  This would not be economically feasible
for a MOOC, where the intention is to offer the course at little or no
cost.  Instead, we must find ways to fill out the lower ranks of the
hierarchy using volunteer labor.  This can be done by identifying the
most capable students and providing them with incentives to help the
others.  In addition, it may be possible to recruit volunteers who
already know the material, perhaps by already having taken the course.
The challenge with this approach is then to devise a suitable
incentive system and to provide the necessary framework for the
volunteers, and the students themselves, to handle most of the load in
performing assessment.

\section{Proposed Work}

We propose creating a peer assessment system that builds on the ideas
for our prototype but overcomes its shortcomings.  In addition, we
propose designing the system to scale from its current capability to
handle large classes to one that could operate in a web-scale course
environment.  Finally, we will carefully analyze and evaluate the
educational value such a system can provide.  In the following we
identify areas requiring further attention and ideas for how they
could be improved.  We also describe possible evaluation approaches.

Just as with our prototype, we plan to deploy and evaluate
the technology as it is being developed within computer science and
mathematics courses at Carnegie Mellon University.  This will give us
access to actual students and undergraduate graders in the context of
actual courses, all at no cost to the research project.

\subsection{Improving Assessment Quality}

Our first goal will be to create an assessment system that provides
reliable and accurate assessments in the context of a large,
undergraduate class.  It assumes that we have a staff of undergraduate
graders, ranging from those who have just taken the class to those who
have several semesters of grading experience and who have gained
mathematical maturity by taking more advanced courses.

Overall, the strategy of breaking the assessment of a proof down into a smaller
set of attributes seems effective, both to make the task manageable by
both student reviewers and graders, and to also enable a uniform
system for assigning scores.  In our current system, however, the set
of attributes is simply tabulated as a flat list.  We have found that
this lack of assessment structure fails to provide enough guidance to the
reviewers and the graders regarding how the different aspects of a
proof should fit together.

\subsubsection*{Pathway verification}

We would like to
create a more structured representation of proofs we call {\em pathway
  verification,}  where the required components of a proof are
explicitly identified.  Some example structures include:
\begin{itemize}
\item A proof by induction must contain an induction hypothesis,
  proofs of the base cases, and proofs of the induction steps.
\item The proof of an if-and-only-if property must have proofs of the
  ``if'' part and the ``only if'' part.
\item A case analysis must include proofs of each possible case.
\end{itemize}
In addition, more complex proofs should be broken down into a set of
lemmas with a valid dependency structure.

For problems assigned in the introductory and intermediate-level
courses we are targeting, there are only be a limited number of
possible pathway structures students can use.  These will be
determined by the instructional staff when devising the assignment and
by grading a sample of 15--25 proofs.  In doing their reviews, the
students will first identify the relevant pathway structure, and then
they will select attributes related to how successfully each component
of the pathway is covered in the proof.  The graders will then refine
the student reviews to devise the precise set of attributes.

We believe that the greater structure provided by pathways will
provide helpful guidance to the students and the graders when doing
their assessments.  It will also provide a useful tool for helping
students better understand how a proof should be structured.  One
could imagine, for example, assigning exercises where the
desired pathway for a proof is given, and the students must
then structure their proofs accordingly.

\subsubsection*{Assigning graders and reviewers}

Our current method of assigning reviewers to proofs attempts to avoid
conflicts of interest, but it otherwise is completely random.  We
believe we could achieve better results by making use of the
heterogeneity of the students, both as proof generators and reviewers,
and the grading staff.  For example, a common strategy in grading
papers is to first look at the ones from the top students to determine
what the best quality results are likely to be.  We can gain some idea
of the likely difficulty of assessing a proof from the prior history
for this student, as well as by some statistical analysis of the
text.  Assuming Zerr's experience that correct proofs are much easier
to grade than incorrect ones, we can bias the assignment of reviewers
and graders so that the more challenging ones are assigned to the
most qualified assessors.

\subsubsection*{Evaluation}

There are multiple methods for measuring the effectiveness of an
assessment technique.  The most straightforward is to compare the
results to those generated by qualified assessors.  This can readily
be done by having a graduate student independently grade a set of
assignments, and then doing a comparison with the assessments
generated by our system.  In addition, we can evaluate the uniformity
of the assessment system by having multiple, independent assessments
of a single set of proofs, and similarly measure the effectiveness of
different assessment systems.

\subsection{Improving Learning Experience}

We can force students to provide peer reviews of each others proofs
by making this an aspect of their course grades, but we would
rather motivate by having it be a useful learning exercise.  We 
want to assign each student a set of proofs that represents a
diversity of possible solutions and that will be appropriate for their
skill level.  In addition, we want their assessment efforts to give
them a better understanding of how a correct proof should be
structured, and the likely ways in which a proof can be invalid.

We believe the learning value of peer assessment can be increased by a
more careful assignment of proofs to student reviewers.  Based on the
prior history of the student writing the proof and by statistical
language analysis, we should be able to estimate properties, such as
1) whether the proof likely to be correct, 2) how clear the presentation
will be, and 3) what pathway structure does it follow.  From this, we
can ensure that students are assigned different types of proofs.
Moreover, we can exploit the heterogeneity of the students and
graders, assigning the most challenging proofs to the people with the
highest abilities.

In addition, we believe that peer reviewers will better internalize
the appropriate proof structures by performing their reviews in the
context of pathway structures.  This will get a detailed view of
alternate ways to prove something and to pinpoint places where
a proof is either invalid or incomplete.

Evaluating the students' perception of the learning benefit of
performing peer reviews can readily be done via surveys.  Evaluating
the actual learning benefit is far more difficult, especially with the
many factors that cannot be controlled, and the small sample size
provided by a single class.  We believe that this aspect of
the assessment system will best be evaluated when we move to
web-scale courses.  Such an environment provides more opportunities to
do A/B testing, where the students are partitioned into different
groups, with different variations of the assessment system applied to
each group.  We can then monitor the performance of the different
student groups in their later assignments.

\section{Future Work}

\subsection{Scaling to Web-Scale Courses}

As we have discussed, a true web-scale course cannot rely on a simple
two-level hierarchy.  In fact, if the course is to be offered at
little or no cost, it cannot even involve a system where assistants
are being paid to directly oversee the efforts of the students.
Instead, we must find a method to recruit volunteer labor, including
current and former students from the course.  In addition to providing
incentives to motivate these volunteers, we must have a very
structured and reliable framework that will ensure that the
assessments are of high quality.

On the other hand, given that current and planned MOOCs do not provide
any form of authenticated credential, the purpose of assessment in
these courses is more to help the student (formative) rather than to
measure performance (summative.)  There is therefore less need for
uniformity and reliability of the assessment and for preventing
collusion among the students and between students and assessors.

We believe that our system for supporting assessments of proofs in
large classes can be scaled to handle web-scale classes.  Typically,
the web-scale course would be based on one that had already been
taught using our system, and so a sufficient set of possible attributes and
pathway structures would already have been generated.


Evaluation: Online course provides an ideal platform for evaluation.
Can do A/B testing.  Have huge amount of digital data.
Scaling to much larger environments

\begin{itemize}
\item Limit effort by instructor and staff to setting up structure and
  testing by grading small set of assignments.
\item Typically, would prototype with class of 200+ students and
  use this to generate set of possible attributes and pathways.
\item Students must be able to reliably assess correctness.
\item Must provide motivation to students to participate and do a good job.
\begin{itemize}
\item Reward verifiers with more verifications of their own work.
\item Identify most skilled verifiers and reward with special status
  or job opportunities.
\end{itemize}
\item Exploit range of verifier capabilities.  Most capable would
  identify cases where new attributes or pathways required.
\end{itemize}

Evaluation of effectiveness

\begin{itemize}
\item Analyze accuracy and consistency of student-generated
  assessments, compared to ones generated by experts and to each other.
\item Measure effectiveness of verification as a learning experience.
  Compare student performance to that in other offerings of course.
  In web-scale environment, could do more A/B testing where different
  students are given different degrees of verification support.
\item Use as data set for understanding how students learn to do proofs.
\item Use as structured methodology for teaching how to write good proofs.
\begin{itemize}
\item Explicit use of pathway structure.
\item Distill into Proof Concept Inventory.
\end{itemize}

\end{itemize}

\subsection{Additional Deployments}

\begin{itemize}
\item Incorporate into 15-122 for evaluating program invariants.  
\item Deploy at some other institution
\begin{itemize}
\item Need large classes to ensure redundancy and anonymity.
\item Need to test on students with range of abilities.
\end{itemize}
\end{itemize}

\subsection{Beyond grading proofs}

\begin{itemize}
\item Use to assess and teach good programming style.
\end{itemize}

\section{Results from Prior NSF Support}

\bibliography{refs}

\end{document}
