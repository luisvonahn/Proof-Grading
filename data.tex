\documentclass[12pt]{article}
\usepackage{fullpage}
\usepackage{times}
\pagestyle{empty}
\setlength{\footskip}{0em}
\setlength{\textheight}{9in}

\begin{document}
\section*{Data Management Plan}

The data generated by the proposed research includes:
\begin{enumerate}
\item Software implementing an assessment system
\item The results of deploying the assessment system in one or more
  courses, including:
\begin{enumerate}
\item
Assignments created by the instructor
\item Submissions by the students
\item Reviews performed by the students
\item Final assessments created by the grading staff
\item Other grading information
\end{enumerate}
\end{enumerate}

All of these data will be in textual format: computer code (Python,
Javascript) for the software, Latex for the assignments and the
student submissions, and various data files for the assessments and
grading.

Our ability to distribute student data will be determined by our IRB,
but we anticipate that we can provide
the above-listed information to other researchers,
but with all identifying information about the students removed.
This will include such information as major, year, and course or
program performance.

We will maintain information about each course as a set of protected
files on the Andrew File System (AFS)\@.  Assuming IRB approval, we
will store the information in a way that we can disseminate the data
by simply copying and supplying some portion of the course directory
structure as a Unix Tar archive.  The directories will contain a set
of README files documenting the structure and format of the files.
Course directories are typically maintained for many years, and so we
anticipate being able to make this information available for at least
five years.  We do not believe the data would have much research value
beyond this point, because it will be obsolete by then.

\end{document}
