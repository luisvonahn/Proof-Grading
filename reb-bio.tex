\documentclass{article}
\usepackage{times}
\usepackage{fullpage}
\usepackage{latexsym}
\usepackage{amssymb}

\newenvironment{denseenumerate}% like denseenumerate but with less separation
	{\begin{enumerate}\setlength{\itemsep}{0pt}\setlength{\parsep}{0pt}}% 
	{\end{enumerate}}


\begin{document}

\begin{center}
{\large\bf Randal E.~Bryant}
\end{center}

\begin{flushleft}
\begin{tabular}{p{3.75in}l}
Dean and University Professor of Computer Science      & (412) 268-8821 (voice)\\
School of Computer Science & (412) 268-5497 (fax)\\
Carnegie Mellon University            & {\tt Randy.Bryant@cs.cmu.edu} (e-mail)\\
Pittsburgh, PA 15213     & \verb+http://www.cs.cmu.edu/~bryant/+\\
\end{tabular}
\end{flushleft}

\section*{Education}

\begin{tabular}{ll}
1973 B.S., University of Michigan & Applied Mathematics\\
1977 S.M., Massachusetts Institute of Technology & Electrical Engineering\\
1978 E.E., Massachusetts Institute of Technology & Electrical Engineering\\
1981 Ph.D., Massachusetts Institute of Technology & Computer Science\\
\end{tabular}

\section*{Employment Experience}

\begin{tabular}{ll}
2004-- & Dean, School of Computer Science, Carnegie Mellon University\\
2004-- & University Professor, Carnegie Mellon University\\
1999--2004 & Head, Computer Science Department, Carnegie Mellon University\\
1997--2004 & President's Professor, Dept.\ Computer Science, Carnegie Mellon University \\
1992--1997 & Professor, Dept.\ Computer Science, Carnegie Mellon University \\
1990--1991 & Visiting Research Fellow, Fujitsu Laboratories, Kawasaki, Japan \\
1988--1992 & Associate Professor, Dept.\ Computer Science, Carnegie Mellon University \\
1984--1988 & Assistant Professor, Dept.\ Computer Science, Carnegie Mellon University \\
1981--1984 & Assistant Professor of Computer Science, California Institute of Technology\\
\end{tabular}

\section*{Related Publications}

\begin{denseenumerate}
\item
R. E. Bryant, and D. R. O'Hallaron,
{\em Computer Systems: A Programmer's Perspective, Second Edition},
Prentice-Hall, 2011.

\item
R. E. Bryant, and D. R. O'Hallaron,
``Teaching Computer Systems from a Programmer's Perspective,''
{\em 32nd Technical Symposium on Computer Science Education SIGCSE~'01},
February, 2001.
\end{denseenumerate}

\section*{Other Publications}

\begin{denseenumerate}
\item
R. E. Bryant,
``Data-Intensive Scalable Computing for Scientific Applications,''
{\em IEEE Computing in Science and Engineering},
Vol.~13, No.~6 (2011),
pp.~25--33.

\item
B. A. Brady, R. E. Bryant, and S. A. Seshia,
``Learning Conditional Abstractions,''
{\em Formal Methods in Computer-Aided Design},
October, 2011,
pp.~116--124.

\item
R. E. Bryant,
``On the Complexity of VLSI Implementations and Graph Representations of
Boolean Functions with Application to Integer Multiplication,''
{\it IEEE Transactions on Computers}, Vol.~40, No.~2 (February, 1991),
pp.~205--213.

\item
R. E. Bryant,
``Symbolic Boolean Manipulation with Ordered Binary Decision
Diagrams,''
{\em ACM Computing Surveys},
Vol.~24, No.~3 (September, 1992), pp.~293--318.

\item
R. E. Bryant,
``Graph-Based Algorithms for Boolean Function Manipulation,''
 {\it IEEE Transactions on Computers}, Vol.\ C-35, No.\ 8 (August, 1986),
pp.~677--691.

\end{denseenumerate}

\section*{Synergistic Activities}

R. E. Bryant has taught undergraduate courses for over 30 years on a
variety of subjects, including computer architecture,
computer systems, algorithms, and data structures, and with enrollments
ranging from 5 to 350.  

Starting in 1998, he and Prof.~David R. O'Hallaron pioneered a new
approach to teaching introductory computer systems, taking the
perspective of how advanced programmers analyze the behavior and
performance of programs running on computers.  Topics range from data
representations and machine-level programming to memory systems,
concurrency, and networking.  This led to their publication of the
textbook {\em Computer Systems: A Programmer's Perspective}, now in
its second edition and in use at over 130 schools worldwide.

To deal with the large class sizes in their system course, Bryant and
O'Hallaron devised a number of tools and techniques for automatically
generating assignments, providing feedback to students, and grading.
They provide students with code checkers based on formal verification
technology, that will, in effect, exhaustively test their code and
return either a message stating their implementation is correct, or an
error message including arguments that cause a function to fail.  In
others, they use randomization to generate unique problems for each
student.  These problems take the form of puzzles that, in solving,
students learn about machine-level programming and buffer overflow
attacks.

Bryant's main research focus has been in devising methods for formally
verifying the correctness of hardware and software.  As part of this
work, he devised a data structure and associated set of algorithms for
representing and manipulating Boolean functions based on Ordered
Binary Decision Diagrams (OBDDs).  OBDDs are now used for a variety of
tasks, including verification, testing, AI planning, and fault
analysis.  His 1986 paper on OBDDs has one of the highest citations
counts in the computer science literature.

\section*{Recent Collaborators}
\begin{denseenumerate}
\item Brian Brady, IBM
\item Garth Gibson, Carnegie Mellon University
\item David O'Hallaron, Carnegie Mellon University
\item John W. O'Leary, Intel
\end{denseenumerate}

\section*{Thesis Advisees}
\begin{denseenumerate}
\item Sanjit Seshia, U.C., Berkeley
\item Shuvendu Lahiri, Microsoft
\end{denseenumerate}

\section*{Graduate Advisor}
Jack B. Dennis, MIT (emeritus)


%
\end{document}
