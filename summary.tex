\documentclass[12pt]{article}
\usepackage{fullpage}
\usepackage{times}
\pagestyle{empty}
\setlength{\footskip}{0em}
\setlength{\textheight}{9in}

\begin{document}
\section*{Project Summary}

Learning to write clear and precise proofs is a challenging but
important task for students studying mathematics and computer science.
Writing a good proof requires students to internalize a core set of
rules and technical skills, while also understanding how to
communicate concepts to others.

Students learning to write proofs require timely and reliable feedback
so that they can correct their flaws and improve their
presentations.  The current method of providing this feedback, by
having an expert (the instructor or a graduate teaching assistant)
grade their assignments manually, is fundamentally unscalable.  It becomes
problematic for large classes (with enrollments of 200 students or
more), and it is completely out of the question for {\em web-scale}
courses, having 10,000 or more participants.

An alternate approach to assessing proofs (both for providing feedback
and for grading) is to make use of undergraduate students---those
currently enrolled as well as those who have already taken the course
and are working as undergraduate graders.  This {\em peer assessment}
has the advantage that the workforce scales with the number of
students enrolled.  It also provides students the opportunity to
critically review proofs with different levels of quality and
correctness, enhancing their own ability to read and write proofs.
Peer assessment, of course, poses its own challenge: to ensure both
quality and consistency, because the people performing the assessment
have limited mastery of the material and limited practice in
reviewing proofs.

This project will develop and evaluate a system for assessing proofs
in undergraduate mathematics and computer science classes via peer
evaluation.  It approaches the problem as an instance of {\em human
  computation}, using computer technology to harness the collective
capability of large numbers of people to do useful work.  This
requires breaking down the task of assessing a proof to make it
possible for multiple, nonexpert people to contribute to the
assessment.  It requires instituting mechanism to ensure the quality,
uniformity, and integrity of the assessment process.

The work will build on existing experience with a prototype system
that has been used for two years in computer science and
mathematics courses at Carnegie Mellon University The current system
targets handling large classes using teams of undergraduate graders.
Experience to date indicates the feasibility of the basic
strategy, but also many aspects that need to be improved to ensure the
quality of the assessments and of the learning experience for the
assessors.  In addition, the research will take initial steps toward an
assessment system that can scale to support web-scale courses.

For {\bf intellectual merit}, this research will study a new realm of
human computation, harnessing the abilities of students with limited
mathematical training to provide reliable and useful assessments of
proofs.  It will devise structured frameworks in which to analyze
proofs, so that assessment can be decomposed into a number of tasks,
suitable for nonexperts.
It will explore ways to maximize the learning experience
students gain by critically evaluating each other's proofs.  It will
will create a platform for performing quantitative experiments on the
most effective methods for teaching students how to read and evaluate
proofs. 

As {\bf broader impacts}, this research seeks to enable students in
large classes and in web-scale courses to enhance their learning by
receiving useful feedback on proofs they write.  It will bring to
large numbers of students an opportunity that is otherwise available
only in environments consisting of small numbers of students getting
attention from expert mentors.  Only a truly scalable assessment
system can meet the needs of the many millions of people worldwide who
could benefit from high quality mathematics education.
\end{document}
